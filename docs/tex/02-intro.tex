\chapter*{Введение}
\addcontentsline{toc}{chapter}{Введение}

В современном мире компьютерная графика используется достаточно широко. Типичная область ее применения – это кинематография и компьютерные игры.

На сегодняшний день большое внимание уделяется алгоритмам получения реалистичного изображения. Такие алгоритмы являются одними из самых затратных по времени, потому что они должны учитывать множество физических явлений, таких как преломление, отражение, рассеивание света. Для повышения реалистичности изображения также учитывается дифракция, вторичное, троичное отражение света, поглощение. 

Можно заметить, что чем более качественным является изображение на выходе алгоритма, тем больше времени и памяти используется для его синтеза. Это и становится проблемой при создании динамической сцены, так как на каждом временном интервале необходимо производить расчеты заново. 

Целью курсовой работы является изучение работы графического конвейера, способов обработки данных на GPGPU, применимости чрезвычайно параллельных алгоритмов к компьютерной графике, а также программная эмуляция работы графического конвейера на видеокарте. 
Предполагается, что большая часть вычислений будет проводиться на графическом ускорителе. Некоторые шаги, например шаг куллинга, можно вынести на процессор. 

Чтобы достигнуть поставленной цели, требуется решить следующие задачи:

\begin{itemize}
	\item описать структуру графического конвейера;
    \item выбрать способ представления объектов сцены;
    \item проанализировать существующие алгоритмы построения изображения и обосновать выбор тех из них, которые в наибольшей степени подходят для решения поставленной задачи;
    \item реализовать выбранные алгоритмы;
    \item разработать программное обеспечение для отображения сцены; 
    \item провести анализ производительности работы программы в зависимости от конфигурации сцены и доступности ресурсов.
\end{itemize}
